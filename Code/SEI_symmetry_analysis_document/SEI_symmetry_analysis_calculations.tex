% -----------------------------------------------------------------------
% DEFINE DOCUMENT
% -----------------------------------------------------------------------
% Define if you want an article, beamer, a book, a letter etc.
\documentclass[12pt]{article}
% -----------------------------------------------------------------------
% LOAD PACKAGES
% -----------------------------------------------------------------------
\usepackage[utf8]{inputenc} % Standard writing
% General document formatting
\usepackage[margin=1in]{geometry}
% \usepackage[parfill]{parskip} % To skip indentation
\usepackage[utf8]{inputenc}
% Related to math
\usepackage{amsmath,amssymb,amsfonts,amsthm}
% Some more mathematical symbols
\usepackage{mathtools}
% For referencing footnotes
\usepackage[symbol]{footmisc}
% In order to write derivatives quickly and nicely
\usepackage{physics}
% In order to include figures
\usepackage{graphicx}
% To write an algorithm
\usepackage[ruled,vlined]{algorithm2e}
% To add todo lists
\usepackage[colorinlistoftodos]{todonotes}
% -----------------------------------------------------------------------
% GENERAL FORMATTING
% -----------------------------------------------------------------------
% Setting length of the margins
\setlength {\marginparwidth }{2cm}
% For mathematics
\theoremstyle{plain}
\newtheorem{theorem}{Theorem}
\theoremstyle{definition}
\newtheorem{definition}[theorem]{Definition}
\newtheorem{example}[theorem]{Example}
% To define the footnotes
\renewcommand{\thefootnote}{\fnsymbol{footnote}}
% Define inputs and outputs for the algorithm
\SetKwInput{KwInput}{Input}
\SetKwInput{KwOutput}{Output}
% Set the graphics-path, i.e. where we store the figures
%\graphicspath{{./images/}}
% -----------------------------------------------------------------------
% DOCUMENT BEGINS
% -----------------------------------------------------------------------
\begin{document}
\title{A symmetry-based structural SI analysis of the SEI model}
\author{Johannes G. Borgqvist}
\date{\today}
\maketitle

ODE for $S$:\begin{equation}\frac{d}{d t} S{\left(t \right)} = - \beta I{\left(t \right)} S{\left(t \right)} + c - \mu_{S} S{\left(t \right)}\end{equation}ODE for $E$:\begin{equation}\frac{d}{d t} E{\left(t \right)} = \beta \left(1 - \epsilon\right) I{\left(t \right)} S{\left(t \right)} - \delta E{\left(t \right)} - \mu_{E} E{\left(t \right)}\end{equation}ODE for $I$:\begin{equation}\frac{d}{d t} I{\left(t \right)} = \beta \epsilon I{\left(t \right)} S{\left(t \right)} + \delta E{\left(t \right)} - \mu_{I} I{\left(t \right)}\end{equation}

% -----------------------------------------------------------------------
% DOCUMENT ENDS
% -----------------------------------------------------------------------
\end{document}
