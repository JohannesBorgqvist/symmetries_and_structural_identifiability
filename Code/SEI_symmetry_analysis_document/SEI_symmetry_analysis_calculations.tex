% -----------------------------------------------------------------------
% DEFINE DOCUMENT
% -----------------------------------------------------------------------
% Define if you want an article, beamer, a book, a letter etc.
\documentclass[12pt]{article}
% -----------------------------------------------------------------------
% LOAD PACKAGES
% -----------------------------------------------------------------------
\usepackage[utf8]{inputenc} % Standard writing
% General document formatting
\usepackage[margin=1in]{geometry}
% \usepackage[parfill]{parskip} % To skip indentation
\usepackage[utf8]{inputenc}
% Related to math
\usepackage{amsmath,amssymb,amsfonts,amsthm}
% Some more mathematical symbols
\usepackage{mathtools}
% For referencing footnotes
\usepackage[symbol]{footmisc}
% In order to write derivatives quickly and nicely
\usepackage{physics}
% In order to include figures
\usepackage{graphicx}
% To write an algorithm
\usepackage[ruled,vlined]{algorithm2e}
% To add todo lists
\usepackage[colorinlistoftodos]{todonotes}
% -----------------------------------------------------------------------
% GENERAL FORMATTING
% -----------------------------------------------------------------------
% Setting length of the margins
\setlength {\marginparwidth }{2cm}
% For mathematics
\theoremstyle{plain}
\newtheorem{theorem}{Theorem}
\theoremstyle{definition}
\newtheorem{definition}[theorem]{Definition}
\newtheorem{example}[theorem]{Example}
% To define the footnotes
\renewcommand{\thefootnote}{\fnsymbol{footnote}}
% Define inputs and outputs for the algorithm
\SetKwInput{KwInput}{Input}
\SetKwInput{KwOutput}{Output}
% Set the graphics-path, i.e. where we store the figures
%\graphicspath{{./images/}}
% -----------------------------------------------------------------------
% DOCUMENT BEGINS
% -----------------------------------------------------------------------
\begin{document}
\title{A symmetry-based structural SI analysis of the SEI model}
\author{Johannes G. Borgqvist}
\date{\today}
\maketitle
% -----------------------------------------------------------------------
% Automated notes generated from SymPy
% -----------------------------------------------------------------------
%\section{Automated notes generated by the SymPy-based script}
%ODE for $S$:\begin{equation}\frac{d}{d t} S{\left(t \right)} = - \beta I{\left(t \right)} S{\left(t \right)} + c - \mu_{S} S{\left(t \right)}\end{equation}ODE for $E$:\begin{equation}\frac{d}{d t} E{\left(t \right)} = \beta \left(1 - \epsilon\right) I{\left(t \right)} S{\left(t \right)} - \delta E{\left(t \right)} - \mu_{E} E{\left(t \right)}\end{equation}ODE for $I$:\begin{equation}\frac{d}{d t} I{\left(t \right)} = \beta \epsilon I{\left(t \right)} S{\left(t \right)} + \delta E{\left(t \right)} - \mu_{I} I{\left(t \right)}\end{equation}
% -----------------------------------------------------------------------
% Nicely typed notes in a coherent text
% -----------------------------------------------------------------------
\section{Symmetry-based local SI analysis of the SEI model}
We consider the following SEI model of epidemiological transmission of tuberculosis:
\begin{align}
  \dot{S} &= - \beta I S + c - \mu_{S} S\,,\label{eq:ODE_S}\\
  \dot{E} &= \beta \left(1 - \upsilon\right) I S - \delta E - \mu_{E} E\,,\label{eq:ODE_E}\\
  \dot{I} &= \beta \upsilon I S + \delta E - \mu_{I} I\,.\label{eq:ODE_I}
\end{align}
We also observe the following two outputs
\begin{align}
  y_{E}&=k_{E}E\,,\label{eq:output_E}\\
  y_{I}&=k_{I}I\,,\label{eq:output_I}  
\end{align}
and their interpretation is that we observe proportions $k_{E}$ and $k_{I}$ of the exposed and infected populations, respectively. In total, we have nine parameters collected in the vector $\boldsymbol{\theta}\in\mathbb{R}^{9}$ which are given by
\begin{equation}
\boldsymbol{\theta}=\begin{pmatrix}c\\\beta\\\mu_{S}\\\mu_{E}\\\mu_{I}\\\delta\\\upsilon\\k_{E}\\k_{I}\end{pmatrix}\,.
  \label{eq:param_SEI}
\end{equation}
We are looking for a family of infinitesimal generators of the Lie group given by
\begin{equation}
  \begin{split}
    X&=\xi(t,S,E,I,\boldsymbol{\theta})\partial_{t}+\eta_{S}(t,S,E,I,\boldsymbol{\theta})\partial_{t}+\eta_{E}(t,S,E,I,\boldsymbol{\theta})\partial_{t}+\eta_{I}(t,S,E,I,\boldsymbol{\theta})\partial_{t}\\
    &\quad+\chi_{c}(\boldsymbol{\theta})\partial_{c}+\chi_{\beta}(\boldsymbol{\theta})\partial_{\beta}+\chi_{\mu_{S}}(\boldsymbol{\theta})\partial_{\mu_{S}}+\chi_{\mu_{E}}(\boldsymbol{\theta})\partial_{\mu_{E}}+\chi_{\mu_{I}}(\boldsymbol{\theta})\partial_{\mu_{I}}+\chi_{\delta}(\boldsymbol{\theta})\partial_{\mu_{\delta}}\\
    &\quad+\chi_{\upsilon}(\boldsymbol{\theta})\partial_{\mu_{\upsilon}}+\chi_{k_{E}}(\boldsymbol{\theta})\partial_{k_{E}}+\chi_{k_{I}}(\boldsymbol{\theta})\partial_{k_{I}}\,.
    \end{split}
  \label{eq:X_SEI_original}
\end{equation}
The three first prolongations are given by
\begin{align}
  \eta_{S}^{(1)}(t,S,E,I,\dot{S},\boldsymbol{\theta})&=D_{t}\eta_{S}(t,S,E,I,\boldsymbol{\theta})-\dot{S}D_{t}\xi(t,S,E,I,\boldsymbol{\theta})\label{eq:eta_S_1}\,,\\
  \eta_{E}^{(1)}(t,S,E,I,\dot{E},\boldsymbol{\theta})&=D_{t}\eta_{E}(t,S,E,I,\boldsymbol{\theta})-\dot{E}D_{t}\xi(t,S,E,I,\boldsymbol{\theta})\label{eq:eta_E_1}\,,\\
\eta_{I}^{(1)}(t,S,E,I,\dot{S},\boldsymbol{\theta})&=D_{t}\eta_{I}(t,S,E,I,\boldsymbol{\theta})-\dot{I}D_{t}\xi(t,S,E,I,\boldsymbol{\theta})\label{eq:eta_I_1}\,,
\end{align}
where the total derivative is defined by: $D_{t}=\partial_{t}+\dot{S}\partial_{S}+\dot{E}\partial_{E}+\dot{I}\partial_{I}$. These prolongations define the first prolongation of the infinitesimal generator $X^{(1)}$ according to
\begin{equation}
  X^{(1)}=X+\eta_{S}^{(1)}\partial_{\dot{S}}+\eta_{E}^{(1)}\partial_{\dot{E}}+\eta_{I}^{(1)}\partial_{\dot{I}}\,.
  \label{eq:X_1_SEI}
\end{equation}
Before, we define the linearised symmetry conditions, we make two critical simplifications. First, the model of interest is autonomous which implies that the time infinitesimal is a constant, i.e. $\xi(t,S,E,I)=K$ for some $K\in\mathbb{R}$ and thus $D_{t}\xi=0$.

Second, the fact that the observed outputs are differential invariants of our generator yields equations for the infinitesimals $\eta_{E}$ and $\eta_{I}$, respectively. Starting with the infinitesimal for $E$, we have that
$$X(y_{E})=0\Longrightarrow\quad{k_{E}}\eta_{E}+\chi_{K_{E}}E=0\,,$$
which gives us the following equation for $\eta_{E}$
\begin{equation}
  \eta_{E}=-\left(\dfrac{\chi_{k_{E}}}{k_{E}}\right)E\,.
  \label{eq:eta_E}
\end{equation}
Analogously, the equation for $\eta_{I}$ is given by
\begin{equation}
  \eta_{I}=-\left(\dfrac{\chi_{k_{I}}}{k_{I}}\right)I\,.
  \label{eq:eta_I}
\end{equation}
Thus, the corresponding prolongations simplify to
\begin{align}
\eta_{E}^{(1)}&=-\left(\dfrac{\chi_{k_{E}}}{k_{E}}\right)\dot{E}\,.\label{eq:eta_E_prolong}\\
  \eta_{I}^{(1)}&=-\left(\dfrac{\chi_{k_{I}}}{k_{I}}\right)\dot{I}\,.\label{eq:eta_I_prolong}
\end{align}
Given these simplifying assumptions and conditions, we can now assemble the linearised symmetry conditions.

Starting with the linearised symmetry condition for $E$, it is given by
\begin{equation}
  \begin{split}
    - \left(\beta \left(1 - \upsilon\right) I S - (\delta + \mu_{E}) E\right) \left(\dfrac{\chi_{k_{E}}}{k_{E}}\right) &= \beta \left(1 - \upsilon\right) I \eta_{S} - \beta I S \chi_{\upsilon} - \beta \left(1 - \upsilon\right) I S \left(\dfrac{\chi_{k_{I}}}{k_{I}}\right)\\
    &\quad+ \left(1 - \upsilon\right) I S \chi_{\beta} - E (\chi_{\delta} + \chi_{\mu_{E}}) + \left(\delta + \mu_{E}\right) E \left(\dfrac{\chi_{k_{E}}}{k_{E}}\right)\,.
  \end{split}
  \label{eq:lin_sym_E}
\end{equation}
Similarly, the linearised symmetry condition for $I$ is given by
\begin{equation}
  \begin{split}
    - \left(\beta \upsilon I S + \delta E - \mu_{I} I\right) \left(\dfrac{\chi_{k_{I}}}{k_{I}}\right) &= \beta \upsilon I \eta_{S} + \beta I S \chi_{\upsilon} - \delta E \left(\dfrac{\chi_{k_{E}}}{k_{E}}\right) + \upsilon I S \chi_{\beta} + E \chi_{\delta}\\
    &\quad- I \chi_{\mu_{I}} - \left(\beta \upsilon S - \mu_{I}\right) I \left(\dfrac{\chi_{k_{I}}}{k_{I}}\right)\,.
\end{split}
\label{eq:lin_sym_I}
\end{equation}
Next, we solve the linearised symmetry conditions for $E$ and $I$ in Eqs. \eqref{eq:lin_sym_E} and \eqref{eq:lin_sym_I}, respectively, for $\eta_S$ and equate them. After some simplifications, we obtain the following linearised symmetry condition

\begin{equation}
\begin{split}
  &\quad- \beta k_{E} k_{I} I S \chi_{\upsilon} + \beta k_{E} \upsilon^{2} I S \chi_{k_{I}} - \beta k_{E} \upsilon I S \chi_{k_{I}} - \beta k_{I} \upsilon^{2} I S \chi_{k_{E}} + \beta k_{I} \upsilon I S \chi_{k_{E}} \\
  &\quad+ \delta k_{E} \upsilon E \chi_{k_{I}} - \delta k_{E} E \chi_{k_{I}} - \delta k_{I} \upsilon E \chi_{k_{E}} + \delta k_{I} E \chi_{k_{E}} - k_{E} k_{I} \upsilon E \chi_{\mu_{E}}\\
  &\quad- k_{E} k_{I} \upsilon I \chi_{\mu_{I}} - k_{E} k_{I} E \chi_{\delta} + k_{E} k_{I} I \chi_{\mu_{I}} = 0\,.
  \end{split}
  \label{eq:eta_S_equality}
\end{equation}
This equation depend on six unknown infinitesimals which we collect in the vector $\boldsymbol{\chi}_{0}\in\mathbb{R}^{6}$ given by
\begin{equation}
  \boldsymbol{\chi}_{0}=\begin{pmatrix}\chi_{\upsilon}\\\chi_{k_{I}}\\\chi_{k_{E}}\\\chi_{\mu_{E}}\\\chi_{\mu_{I}}\\\chi_{\delta}\end{pmatrix}\,.
  \label{eq:chi_vec_sub}
\end{equation}
Next, we split up Eq. \eqref{eq:eta_S_equality} with respect to iterated monomials of $\{S,E,I\}$. This procedure yields the following linear equations.


\begin{align}
I:&\quad- k_{E} k_{I} \upsilon \chi_{\mu_{I}} + k_{E} k_{I} \chi_{\mu_{I}}&=0\,,\label{eq:det_eq_0}\\
E:&\quad\delta k_{E} \upsilon \chi_{k_{I}} - \delta k_{E} \chi_{k_{I}} - \delta k_{I} \upsilon \chi_{k_{E}} + \delta k_{I} \chi_{k_{E}} - k_{E} k_{I} \upsilon \chi_{\mu_{E}} - k_{E} k_{I} \chi_{\delta}&=0\,,\label{eq:det_eq_1}\\
I S:&\quad- \beta k_{E} k_{I} \chi_{\upsilon} + \beta k_{E} \upsilon^{2} \chi_{k_{I}} - \beta k_{E} \upsilon \chi_{k_{I}} - \beta k_{I} \upsilon^{2} \chi_{k_{E}} + \beta kI \upsilon \chi_{k_{E}}&=0\,.\label{eq:det_eq_2}
\end{align}
The above linear system of equations can be expressed as the following matrix system $M\boldsymbol{\chi_{0}}=\boldsymbol{0}$
\begin{equation}
  \begin{split}
    &\underset{=M}{\underbrace{\begin{pmatrix}0 & 0 & 0 & 0 & - k_{E} k_{I} \upsilon + k_{E} k_{I} & 0\\0 & \delta k_{E} \upsilon - \delta k_{E} & - \delta k_{I} \upsilon + \delta k_{I} & - k_{E} k_{I} \upsilon & 0 & - k_{E} k_{I}\\- \beta k_{E} k_{I} & \beta k_{E} \upsilon^{2} - \beta k_{E} \upsilon & - \beta k_{I} \upsilon^{2} + \beta k_{I} \upsilon & 0 & 0 & 0\end{pmatrix}}}\underset{=\boldsymbol{\chi}_{0}}{\underbrace{\begin{pmatrix}\chi_{\upsilon}\\\chi_{k_{I}}\\\chi_{k_{E}}\\\chi_{\mu_{E}}\\\chi_{\mu_{I}}\\\chi_{\delta}\end{pmatrix}}}\\
    &=\underset{=\mathbf{0}}{\underbrace{\begin{pmatrix}0\\0\\0\end{pmatrix}}}\,,
   \end{split}
  \label{eq:mat_sys_1}
\end{equation}
where $M\in\mathbb{R}^{3\times{6}}$, $\boldsymbol{\chi}_{0}\in\mathbb{R}^{6}$ and $\mathbf{0}\in\mathbb{R}^{3}$. Clearly, the parameter infinitesimal vector $\boldsymbol{\chi}_{0}\in\mathbb{R}^{6}$ that solves this matrix equation is given by a linear combination of the basis vectors of the nullspace of the matrix $M$ which is denoted by $\mathcal{N}(M)$. This nullspace is three-dimensional and is given by
\begin{equation}
  \mathcal{N}(M)=\left\{ \begin{pmatrix}0\\\frac{k_{I}}{k_{E}}\\1\\0\\0\\0\end{pmatrix}, \  \begin{pmatrix}\frac{\upsilon^{2}}{\delta}\\\frac{k_{I} \upsilon}{\delta \upsilon - \delta}\\0\\1\\0\\0\end{pmatrix}, \  \begin{pmatrix}\frac{\upsilon}{\delta}\\\frac{k_{I}}{\delta \upsilon - \delta}\\0\\0\\0\\1\end{pmatrix}\right\}\,.
  \label{eq:nullspace_chi}
\end{equation}
Given three arbitrary coefficients $\alpha_{1},\alpha_{2},\alpha_{3}\in\mathbb{R}$, we have the following solution for six of the parameter infinitesimals:


\begin{equation}
  \begin{pmatrix}\chi_{\upsilon}\\\chi_{k_{I}}\\\chi_{k_{E}}\\\chi_{\mu_{E}}\\\chi_{\mu_{I}}\\\chi_{\delta}\end{pmatrix}=\begin{pmatrix}\upsilon \left(\alpha_{2} \upsilon + \alpha_{3}\right)\\\frac{k_{I} \left(\alpha_{1} \left(\upsilon - 1\right) + \alpha_{2} \upsilon + \alpha_{3}\right)}{\upsilon - 1}\\\alpha_{1} k_{E}\\\alpha_{2} \delta\\0\\\alpha_{3} \delta\end{pmatrix}\,.
\label{eq:chi_0_sol}
\end{equation}
From these calculations, we see immediately that $\chi_{\mu_{I}}=0$ and hence $\mu_{I}$ is locally structurally identifiable. After substituting these values into the equation for $\eta_{S}$, this infinitesimal is given by
\begin{equation}
  \eta_{S}=- \frac{a_{2} \beta \upsilon^{2} I S}{\beta \upsilon I - \beta I} + \frac{a_{2} \beta \upsilon I S}{\beta \upsilon I - \beta I} - \frac{a_{2} \delta E}{\beta \upsilon I - \beta I} - \frac{a_{3} \beta \upsilon I S}{\beta \upsilon I - \beta I} + \frac{a_{3} \beta I S}{\beta \upsilon I - \beta I} - \frac{a_{3} \delta E}{\beta \upsilon I - \beta I} - \frac{\upsilon I S \chi_{\beta}}{\beta \upsilon I - \beta I} + \frac{I S \chi_{\beta}}{\beta \upsilon I - \beta I}\,.
  \label{eq:eta_S}
\end{equation}
As a result of substituting this infinitesimal into its linearised symmetry condition, and then decomposing the resulting linearised symmetry condition with respect to iterated monomials of $\{S,E,I\}$, we obtain the following linear system of equations for six of the infinitesimals.

\begin{align}
I^{2}:&\quad- \alpha_{2} \beta c \upsilon^{2} + \alpha_{2} \beta c \upsilon - \alpha_{3} \beta c \upsilon + \alpha_{3} \beta c - \beta \upsilon \chi_{c} + \beta \chi_{c} - c \upsilon \chi_{\beta} + c \chi_{\beta}&=0\,,\label{eq:det_eq_0}\\
E I:&\quad\alpha_{2} \delta^{2} + \alpha_{2} \delta \mu_{E} - \alpha_{2} \delta \mu_{I} - \alpha_{2} \delta \mu_{S} + \alpha_{3} \delta^{2} + \alpha_{3} \delta \mu_{E} - \alpha_{3} \delta \mu_{I} - \alpha_{3} \delta \mu_{S}&=0\,,\label{eq:det_eq_1}\\
E^{2}:&\quad\alpha_{2} \delta^{2} + \alpha_{3} \delta^{2}&=0\,,\label{eq:det_eq_2}\\
E I^{2}:&\quad- \alpha_{2} \beta \delta - \alpha_{3} \beta \delta&=0\,,\label{eq:det_eq_3}\\
I^{2} S:&\quad\alpha_{2} \beta \delta \upsilon - \alpha_{2} \beta \delta + \alpha_{3} \beta \delta \upsilon - \alpha_{3} \beta \delta + \beta \upsilon \chi_{\mu_{S}} - \beta \chi_{\mu_{S}}&=0\,,\label{eq:det_eq_4}\\
E I S:&\quad\alpha_{2} \beta \delta \upsilon + \alpha_{3} \beta \delta \upsilon&=0\,,\label{eq:det_eq_5}\\
I^{3} S:&\quad- \alpha_{1} \beta^{2} \upsilon + \alpha_{1} \beta^{2} - \alpha_{2} \beta^{2} \upsilon - \alpha_{3} \beta^{2} + \beta \upsilon \chi_{\beta} - \beta \chi_{\beta}&=0\,.\label{eq:det_eq_6}
\end{align}
We some from Eqs. \eqref{eq:det_eq_2}, \eqref{eq:det_eq_3} and \eqref{eq:det_eq_5} that
\begin{equation}
  \alpha_{2}=-\alpha_{3}\,.
  \label{eq:alpha_eq}
\end{equation}
By substituting this value, the above system is reduced to

\begin{align}
I^{2}:&\quad- \alpha_{2} \beta c \upsilon^{2} + 2 \alpha_{2} \beta c \upsilon - \alpha_{2} \beta c - \beta \upsilon \chi_{c} + \beta \chi_{c} - c \upsilon \chi_{\beta} + c \chi_{\beta}&=0\,,\label{eq:det_eq_new_0}\\
I^{2} S:&\quad\beta \left(\upsilon - 1\right) \chi_{\mu_{S}}&=0\,,\label{eq:det_eq_new_1}\\
I^{3} S:&\quad\beta \left(- \alpha_{1} \beta \upsilon + \alpha_{1} \beta - \alpha_{2} \beta \upsilon + \alpha_{2} \beta + \upsilon \chi_{\beta} - \chi_{\beta}\right)&=0\,.\label{eq:det_eq_new_2}\\
\end{align}
From Eq. \eqref{eq:det_eq_new_1}, we see that
\begin{equation}
  \chi_{\mu_{S}}=0\,,
  \label{eq:chi_muS}
\end{equation}
and hence $\mu_{S}$ is locally structurally identifiable. Next, we solve Eq. \eqref{eq:det_eq_new_2} for $\chi_{\beta}$ which yields
\begin{equation}
  \chi_{\beta} = \beta \left(\alpha_{1} + \alpha_{2}\right)\,.
\label{eq:chi_beta}
\end{equation}
Lastly, we substitute the value for $\chi_{\beta}$ in Eq. \eqref{eq:chi_beta} into Eq. \eqref{eq:det_eq_new_0}, and solve the resulting equation for $\chi_{c}$ resulting in the following parameter infinitesimal
\begin{equation}
  \chi_{c} = - c \left(\alpha_{1} + \alpha_{2} \upsilon\right)\,.
  \label{eq:chi_c}
\end{equation}
All of these calculations result in the following vector for the parameter infinitesimals:
\begin{equation}
\begin{pmatrix}\chi_{c}\\\chi_{\beta}\\\chi_{\mu_{S}}\\\chi_{\mu_{E}}\\\chi_{\mu_{I}}\\\chi_{\delta}\\\chi_{\upsilon}\\\chi_{k_{E}}\\\chi_{k_{I}}\end{pmatrix}=\begin{pmatrix}- c \left(\alpha_{1} + \alpha_{2} \upsilon\right)\\\beta \left(\alpha_{1} + \alpha_{2}\right)\\0\\\alpha_{2} \delta\\0\\- \alpha_{2} \delta\\\alpha_{2} \upsilon \left(\upsilon - 1\right)\\\alpha_{1} k_{E}\\k_{I} \left(\alpha_{1} + \alpha_{2}\right)\end{pmatrix}\,,\label{eq:parameter_infinitesimals}
\end{equation}
which depend on the two arbitrary parameters $\alpha_{1}$ and $\alpha_{2}$. Moreover, this yields the following three infinitesimals for the states $S$, $E$ and $I$:
\begin{align}
  \eta_{S}&=-\left(\alpha_{1} + \alpha_{2} \upsilon\right) S,\label{eq:eta_S_final}\\
  \eta_{E}&=-\alpha_{1} E,\label{eq:eta_E_final}\\
\eta_{I}&=-(\alpha_{1}+\alpha_{2}) I.\label{eq:eta_I_final}  
\end{align}

In summary, this gives us the following family of infinitesimal generators of the Lie group:

\begin{equation}
  \begin{split}
    X &= -\left(f_{1}(\boldsymbol{\theta}) + f_{2}(\boldsymbol{\theta}) \upsilon\right) S\partial_{S} -f_{1}(\boldsymbol{\theta}) E\partial_{E}-(f_{1}(\boldsymbol{\theta})+f_{2}(\boldsymbol{\theta})) I\partial_{I}\\
    &- c \left(f_{1}(\boldsymbol{\theta}) + f_{2}(\boldsymbol{\theta}) \upsilon\right)\partial_{c}+\beta \left(f_{1}(\boldsymbol{\theta}) + f_{2}(\boldsymbol{\theta})\right)\partial_{\beta}+f_{2}(\boldsymbol{\theta}) \delta\partial_{\mu_{E}}- f_{2}(\boldsymbol{\theta}) \delta\partial_{\delta}\\
    &+f_{2}(\boldsymbol{\theta}) \upsilon \left(\upsilon - 1\right)\partial_{\upsilon}+f_{1}(\boldsymbol{\theta})k_{E}\partial_{k_{E}}+(f_{1}(\boldsymbol{\theta})+f_{2}(\boldsymbol{\theta}))k_{I}\partial_{k_{I}}\,,
\end{split}
  \label{eq:Lie_final}
\end{equation}
where we introduced the two functions $f_{1},f_{2}:\mathbb{R}^{p}\mapsto\mathbb{R}$ defined so that $\alpha_{1}=f_{1}(\boldsymbol{\theta})$ and $\alpha_{2}=f_{2}(\boldsymbol{\theta})$. 

% -----------------------------------------------------------------------
% Invariant calculations
% -----------------------------------------------------------------------
\section{Calculation of universal differential invariants}
The universal invariants above are independent of the arbitrary functions $f_{1}(\boldsymbol{\theta})$ and $f_{2}(\boldsymbol{\theta})$ respectively. Using the method of characteristics, we can integrate ODEs which are independent of these two functions and whose right hand sides are defined by the infinitesimals. The resulting integration constants correspond to the universal, since they are independent of the arbitrary functions $f_{1}(\boldsymbol{\theta})$ and $f_{2}(\boldsymbol{\theta})$, invariants which are the locally structurally identifiable and observable quantities.  

First, we see that none of the three states are locally structurally observable. But we can readily find observable quantitites depending on the states by calculating universal differential invariants. For the state $S$, we see that
\begin{equation}
  \dfrac{\mathrm{d}S}{\mathrm{d}c}=\dfrac{S}{c}\,,
  \label{eq:inv_S_eq}
\end{equation}
which is readily integrated to the following differential invariant
\begin{equation}
  I_{1}=\dfrac{S}{c}\,.
  \label{eq:inv_S}
\end{equation}
For the state $E$, we see that 
\begin{equation}
  \dfrac{\mathrm{d}E}{\mathrm{d}k_{E}}=-\dfrac{E}{k_{E}}\,,
  \label{eq:inv_E_eq}
\end{equation}
which is readily integrated to the following differential invariant
\begin{equation}
  I_{2}=k_{E}E\,.
  \label{eq:inv_E}
\end{equation}
Similarly, for the state $I$, we see that 
\begin{equation}
  \dfrac{\mathrm{d}I}{\mathrm{d}k_{I}}=-\dfrac{I}{k_{I}}\,,
  \label{eq:inv_I_eq}
\end{equation}
which is readily integrated to the following differential invariant
\begin{equation}
  I_{3}=k_{I}I\,.
  \label{eq:inv_I}
\end{equation}
Using the same approach, we next calculate the universal parameter invariants.

Since $\chi_{\mu_{S}}=0$ and $\chi_{\mu_{I}}=0$, the parameters $I_{4}=\mu_{S}$ and $I_{5}=\mu_{I}$ are locally structurally identifiable. Moreover, from the infinitesimals $\chi_{\mu_{E}}$ and $\chi_{\delta}$, we see that
\begin{equation}
  \dfrac{\mathrm{d}\mu_{E}}{\mathrm{d}\delta}=-1\,,
  \label{eq:inv_muE_delta_eq}
\end{equation}
which is integrated to give the following universal parameter invariant
\begin{equation}
  I_{6}=\mu_{E}+\delta\,.
  \label{eq:inv_muE_delta}
\end{equation}
From the infinitesimals $\chi_{k_{I}}$ and $\chi_{\beta}$, we see that  
\begin{equation}
  \dfrac{\mathrm{d}k_{I}}{\mathrm{d}\beta}=\dfrac{k_{I}}{\beta}\,,
  \label{eq:inv_muE_delta_eq}
\end{equation}
which is integrated to give the following universal parameter invariant
\begin{equation}
  I_{7}=\dfrac{k_{I}}{\beta}\,.
  \label{eq:inv_kI_beta}
\end{equation}
From the infinitesimals for $\chi_{\upsilon}$ and $\chi_{\delta}$, we see that  
\begin{equation}
  \dfrac{\mathrm{d}\delta}{\mathrm{d}\upsilon}=\dfrac{\delta}{\upsilon(1-\upsilon)}\,,
  \label{eq:inv_delta_upsilon_eq}
\end{equation}
which is integrated to give the following universal parameter invariant
\begin{equation}
  I_{8}=\delta\left(\dfrac{1-\upsilon}{\upsilon}\right)\,.
  \label{eq:inv_kI_beta}
\end{equation}
Next, we consider the combined infinitesimal for the product $c\upsilon$ which is given by
\begin{equation}
  \chi_{c\upsilon}=c\chi_{\upsilon}+\upsilon\chi_{c}=-(\alpha_{1}+\alpha_{2})c\upsilon\,,
  \label{eq:chi_product}
\end{equation}
and this product infinitesimal can be combined with $\chi_{\beta}$ which results in the equation
\begin{equation}
  \dfrac{\mathrm{d}(c\upsilon)}{\mathrm{d}\beta}=-\dfrac{(c\upsilon)}{\beta}\,.
  \label{eq:inv_c_upsilon_beta_eq}
\end{equation}
Integrating this equation yields the following universal parameter invariant
\begin{equation}
  I_{9}=\beta{c}\upsilon\,.
  \label{eq:inv_c_upsilon_beta}
\end{equation}
Finally, we consider the combined infinitesimal for the quotient $k_{E}/\delta$ which is given by
\begin{equation}
  \chi_{k_{E}/\delta}=\dfrac{\delta\chi_{k_{E}}-k_{E}\chi_{\delta}}{\delta^{2}}=(\alpha_{1}+\alpha_{2})\left(\dfrac{k_{E}}{\delta}\right)\,.
  \label{eq:kE_quotient}
\end{equation}
From this quotient infinitesimal and $\chi_{\beta}$, we see that
\begin{equation}
  \dfrac{\mathrm{d}\beta}{\mathrm{d}(k_{E}/\delta)}=\dfrac{\beta}{(k_{E}/\delta)}\,,
  \label{eq:inv_kE_delta_beta_eq}
\end{equation}
which is integrated to give the last universal parameter invariant
\begin{equation}
  I_{10}=\dfrac{\beta\delta}{k_{E}}\,.
  \label{eq:inv_kE_delta_beta}
\end{equation}
% -----------------------------------------------------------------------
% DOCUMENT ENDS
% -----------------------------------------------------------------------
\end{document}
